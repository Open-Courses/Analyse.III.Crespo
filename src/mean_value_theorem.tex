\chapter{Inégalité des accroissements finis (de la moyenne)}

\section*{Motivation et rappel d'analyse réel}

\begin{theorem} [des accroissements finis]
	\label{theorem:mean_value_theorem_real_rappel}
	Soient $a, b \in \real$ tel que $a < b$.
	Soit $\GSfunction{f}{[a, b]}{\real}$ tel que:

	\begin{itemize}
		\item $f$ est continue sur $[a, b]$
		\item $f$ est dérivable sur $]a, b[$.
	\end{itemize}

	Alors il existe $c \in ]a, b[$ tel que $f(b) - f(a) = f'(c) (b - a)$.
\end{theorem}

\ifdefined\outputproof
\begin{proof}

\end{proof}
\fi

\begin{proposition}
	Supposons en plus que $\abs{f'(x)} \leq k$ pour tout $x \in ]a, b[$.
	Alors $\abs{f(b) - f(a)} \leq k \abs{b - a}$.
\end{proposition}

\ifdefined\outputproof
\begin{proof}

\end{proof}
\fi

Remarquons que le théorème des accroissements finis n'est pas vrai dans le cas
vectoriel. En effet, $\GSfunction{f}{[0, 2\pi]}{\real^{2}} : t \rightarrow
(cos(t), sin(t))$ est un contre exemple.

\section{Cas où $E = \real$}

\begin{theorem}
	Soient $a, b \in \real$ tel que $a < b$, et $F$ un espace vectoriel normé.

	Soient $\GSfunction{f}{[a, b]}{F}$ et $\GSfunction{g}{[a, b]}{\real}$ tel
	que:

	\begin{itemize}
		\item $f$ et $g$ sont continues sur $[a, b]$.
		\item $f$ et $g$ sont dérivables sur $]a, b[$
		\item $\forall x \in \, ]a, b[$, $\GSnormeDef{f'(x)}{F} \leq g'(x)$.
	\end{itemize}

	Alors on a $\GSnormeDef{f(b) - f(a)}{F} \leq \abs{g(b) - g(a)}$.
\end{theorem}

\ifdefined\outputproof
\begin{proof}

\end{proof}
\fi

\begin{corollary}
	Soient $a, b \in \real$ tel que $a < b$ et $F$ un espace vectoriel normé.

	Soit $\GSfunction{f}{[a, b]}{F}$ continues sur $[a, b]$ et dérivable sur
	$]a, b[$.

	Supposons en plus qu'il existe $k > 0$ tel que pour tout $x \in ]a, b[$,
	$\GSnormeDef{f'(x)}{F} \leq k$. Alors:

	\begin{equation*}
		\GSnormeDef{f(b) - f(a)}{F} \leq k (b - a).
	\end{equation*}
\end{corollary}

\ifdefined\outputproof
\begin{proof}

\end{proof}
\fi

\section{Cas où $E$ est un espace vectoriel normé}

Soient $E$ et $F$ des espaces vectoriels normés.
Soit $\mathcal{U}$ un ouvert de $E$.

Rappelons d'abord quelques définitions.

\begin{definition}
	Soient $a, b \in E$. On définit \textbf{le segment d'extrémités $a$ et $b$}
	l'ensemble $\segment{a}{b} := \GSsetDef{(1 - t) a + t b}{t \in [0, 1]}$.
\end{definition}

\begin{definition}
	On dit que $\mathcal{U}$ est \textbf{convexe} si pour tout $a, b \in
	\mathcal{U}$, $\segment{a}{b} \subseteq \mathcal{U}$.
\end{definition}

\begin{theorem} [Inégalité des accroissements finis]
	Soit $\GSfunction{f}{\mathcal{U}}{F}$ une application différentiable en tout
	point de $\mathcal{U}$.

	Soient $a, b \in \mathcal{U}$ tel que $\segment{a}{b} \subseteq \mathcal{U}$.

	Alors:

	\begin{equation*}
		\GSnormeDef{f(b) - f(a)}{F} \leq \displaystyle \sup_{x \in \segment{a}{b}}
		\GSnorme{df(x)} \GSnormeDef{b - a}{E}.
	\end{equation*}

	\label{theorem:mean_value_theorem}
\end{theorem}

\ifdefined\outputproof
\begin{proof}

\end{proof}
\fi

\begin{remarque}
	\begin{itemize}
		\item Si $\mathcal{U}$ est convexe, alors on a:
			\begin{equation*}
				\GSnormeDef{f(b) - f(a)}{F} \leq \displaystyle \sup_{x \in \mathcal{U}}
				\GSnorme{df(x)} \GSnormeDef{b - a}{E}.
			\end{equation*}
		\item Si la différentielle est nulle en tout point de $\mathcal{U}$ avec
			$\mathcal{U}$ convexe, alors $f$ est constante.
	\end{itemize}
\end{remarque}

\section{Applications du théorème des accroissements finis}

\subsection{Différentiabilité d'une limite d'applications différentiables}

\begin{corollary}
	Soient $E, F$ deux espaces de Banach.
	Soit $\mathcal{U}$ un ouvert convexe de $E$.

	Soit une suite $\GSsequence{f}{n}{\naturel}$ d'applications différentiables
	de $\mathcal{U}$ dans $F$.

	Supposons:
	\begin{enumerate}
		\item qu'il existe $a \in \mathcal{U}$ tel que $(f_{n}(a))_{n \in
			\naturel}$ converge.
		\item la suite ${(df_{n})}_{n \in \naturel}$ converge uniformément sur
			$\mathcal{U}$ vers
			$\GSfunction{g}{\mathcal{U}}{\GScontinueHomo{E}{F}}$.
	\end{enumerate}

	Alors:

	\begin{enumerate}
		\item ${(f_{n})}_{n \in \naturel}$ converge simplement sur tout
			$\mathcal{U}$. La fonction limite est noté $f$.
		\item ${(f_{n})}_{n \in \naturel}$ converge uniformément sur toute
			partie bornée de $\mathcal{U}$.
		\item $f$ est différentiable sur $\mathcal{U}$, et sa différentielle
			vaut $g$.
		\item Si les $f_{n}$ sont continûment différentiables, $f$ également.
	\end{enumerate}
\end{corollary}

\ifdefined\outputproof
\begin{proof}

\end{proof}
\fi

\begin{corollary}
	Sous les mêmes hypothèses, avec $f_{n} = \GSsum{i}{0}{n}{u_{i}}$ où
	$\GSsequence{u}{n}{\naturel}$ est une suite de fonctions différentiables de
	$\mathcal{U}$ dans $F$, on a les mêmes résultats.
\end{corollary}

\ifdefined\outputproof
\begin{proof}

\end{proof}
\fi

% TODO: Remarques Page 24 Sam

\subsection{Différentiabilité et existence des dérivées partielles}

% TODO: remarque page 25

\begin{theorem}
	\label{theorem:partial_derivative_existence}

	Soient $E = \GSprodSet{i}{1}{n}{E}$, et $\mathcal{U} =
	\GSprodSet{i}{1}{n}{\mathcal{U}}$ un ouvert de $E$.
	Soit $\GSfunction{f}{\mathcal{U}}{F}$.
	Alors les assertions suivantes sont équivalentes:

	-- $f$ est de classe $\mathcal{C}^{1}$ sur $\mathcal{U}$.

	-- Les dérivées partielles
	$\GSfunction{\pdv{f}{x_{i}}}{\mathcal{U}}{\GScontinueHomo{E_{i}}{F}}$
	existent et sont continues sur $\mathcal{U}$.
\end{theorem}

\ifdefined\outputproof
\begin{proof}

\end{proof}
\fi
