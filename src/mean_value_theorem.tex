\chapter{Inégalité des accroissements finis (de la moyenne)}

\section*{Motivation et rappel d'analyse réel}

\section{Cas où $E = \real$}

\section{Cas où $E$ est un espace vectoriel normé}

\section{Applications}

\subsection{Différentiabilité d'une limite d'applications différentiables}

\begin{corollary}
	Soit une suite $\GSsequence{f}{n}{\naturel}$ d'applications différentiables d'un ouvert convexe
	$\mathcal{U} \subseteq E$ dans un espace de Banach $F$.

	Supposons:
	\begin{enumerate}
		\item qu'il existe $a \in \mathcal{U}$ tel que $(f_{n}(a))_{n \in
			\naturel}$ converge.
		\item la suite ${(df_{n})}_{n \in \naturel}$ converge uniformément sur
			$\mathcal{U}$ vers
			$\GSfunction{g}{\mathcal{U}}{\GScontinueHomo{E}{F}}$.
	\end{enumerate}

	Alors:

	\begin{enumerate}
		\item ${(f_{n})}_{n \in \naturel}$ converge simplement sur tout
			$\mathcal{U}$. La fonction limite est noté $f$.
		\item ${(f_{n})}_{n \in \naturel}$ converge uniformément sur toute
			partie bornée de $\mathcal{U}$.
		\item $f$ est différentiable sur $\mathcal{U}$, et sa différentielle
			vaut $g$.
		\item Si les $f_{n}$ sont continûment différentiables, $f$ également.
	\end{enumerate}
\end{corollary}

\ifdefined\outputproof
\begin{proof}

\end{proof}
\fi

\begin{corollary}
	Sous les mêmes hypothèses, avec $f_{n} = \GSsum{n}{0}{\infty}{u_{n}}$ où
	$\GSsequence{u}{n}{\naturel}$ est une suite de fonctions différentiables de $\mathcal{U}$ ouvert
	convexe de $E$ dans $F$ de Banach, on a les mêmes résultats.
\end{corollary}

\ifdefined\outputproof
\begin{proof}

\end{proof}
\fi

\subsection{Différentiabilité et existence des dérivées partielles}

\begin{theorem}
\label{theorem_partial_derivative_existence}

	Soit $E = \GSprodSet{i}{1}{n}{E}$, et $\mathcal{U} =
	\GSprodSet{i}{1}{n}{\mathcal{U}}$ un ouvert de $E$.
	Soit $\GSfunction{f}{\mathcal{U}}{F}$.
	Alors les assertions suivantes sont équivalentes:

	-- $f$ est de classe $\mathcal{C}^{1}$ sur $\mathcal{U}$.

	-- Les dérivées partielles
	$\GSfunction{\pdv{f}{x_{i}}}{\mathcal{U}}{\GScontinueHomo{E_{i}}{F}}$
	existent et sont continues sur $\mathcal{U}$.
\end{theorem}

\ifdefined\outputproof
\begin{proof}

\end{proof}
\fi
