\chapter{Inégalité des accroissements finis (de la moyenne)}

\section*{Motivation et rappel d'analyse réel}

\section{Cas où $E = \real$}

\section{Cas où $E$ est un espace vectoriel normé}

\section{Applications}

\subsection{Différentiabilité d'une limite d'applications différentiables}

\begin{corollary}
	Soit une suite $f_{n}$ d'applications différentiables d'un ouvert convexe
	$\mathcal{U} \subseteq E$ dans un espace de Banach $F$.
	
	Supposons:
	\begin{enumerate}
		\item qu'il existe $a \in \mathcal{U}$ tel que $f_{n}(a)$ converge.
		\item la suite ${(df_{n})}_{n \in \naturel}$ converge uniformément sur
			$\mathcal{U}$ vers
			\GSfunction{$g$}{$\mathcal{U}$}{\GScontinueHomo{E}{F}}.
	\end{enumerate}

	Alors:

	\begin{enumerate}
		\item ${(f_{n})}_{n \in \naturel}$ converge simplement sur tout
			$\mathcal{U}$. La fonction limite est noté $f$.
		\item ${(f_{n})}_{n \in \naturel}$ converge uniformément sur toute
			partie bornée de $\mathcal{U}$.
		\item $f$ est différentiable sur $\mathcal{U}$, et sa différentielle
			vaut $g$.
		\item Si les $f_{n}$ sont continûment différentiables, $f$ également.
	\end{enumerate}
\end{corollary}

\begin{proof}
	
\end{proof}

\begin{corollary}
	Sous les mêmes hypothèses, avec $f_{n}$ = \GSsum{n}{0}{\infty}{$u_{n}$} où
	$u_{n}$ est une suite de fonctions différentiables de $\mathcal{U}$ ouvert
	convexe de $E$ dans $F$ de Banach, on a les mêmes résultats.
\end{corollary}

\begin{proof}
	
\end{proof}

\subsection{Différentiabilité et existence des dérivées partielles}

\begin{theorem}
\label{theorem_partial_derivative_existence}
	Une fonction $f$ est différentiable en $a$ ssi ses dérivées partielles
	existent en $a$ et sont continues.
\end{theorem}

\begin{proof}
	
\end{proof}
