\chapter{Normes, applications linéaires et multilinéaires}


Nous commencerons par rappeler la définition d'une norme, de la topologie
associée à une norme et d'équivalence de norme. Nous rappelerons aussi le
théorème de Riesz et de l'équivalence des normes en dimension finie.

Dans un deuxième temps, nous regarderons les applications linéaire et la notion
de continuité.

Pour finir, nous généraliserons aux applications multilinéaires.

\begin{definition}
    Soit E un espace vectoriel, et \GSfunction{N}{E}{$\mathbb{R}$} une fonction.
    N est une norme si elle vérifie :
    \begin{enumerate}
        \item $\forall x \in E$, $N(x) \geq 0$
        \item $\forall x \in E$, $(N(x) = 0 \Rightarrow x = 0)$
        \item $\forall x, y \in E$, $N(x + y) \leq N(x) + N(y)$ (inégalité
            triangulaire)
        \item $\forall \lambda \in \mathbb{K}$, $\forall x \in E$, $N(\lambda x)
            \leq |\lambda| N(x)$
    \end{enumerate}
    Cette application est souvent dénotée par \GSnormeDef{.}{E} ou tout simplement
    \GSnorme{.}
\end{definition}

\begin{exercice}
    Montrer qu'on a
    $\forall \lambda \in \mathbb{K}$, $\forall x \in E$, $N(\lambda x) =
    |\lambda| N(x)$
\end{exercice}

\begin{definition}
    Soient deux normes \GSnormeDef{.}{1} \GSnormeDef{.}{2} sur E. On dit qu'elles
    sont équivalentes si $\forall x \in E$, $\exists k_{1}, k_{2} \in
    \mathbb{K}$ tel que $k_{1}$\GSnormeDef{x}{1}$\leq$\GSnormeDef{x}{2}$\leq$
    $k_{2}$\GSnormeDef{x}{1}
\end{definition}

