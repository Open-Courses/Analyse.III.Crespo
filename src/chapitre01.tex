\chapter{Normes, applications linéaires et multilinéaires}

Nous commencerons par rappeler la définition d'une norme, de la topologie
associée à une norme et d'équivalence de norme. Nous rappelerons aussi le
théorème de Riesz et de l'équivalence des normes en dimension finie.

Dans un deuxième temps, nous regarderons les applications linéaire et la notion
de continuité.

Pour finir, nous généraliserons aux applications multilinéaires.

\section{Normes et espace de Banach}

\begin{definition}
    Soit E un espace vectoriel, et \GSfunction{N}{E}{$\mathbb{R}$} une fonction.
    N est une norme si elle vérifie :
    \begin{enumerate}
        \item $\forall x \in E$, $N(x) \geq 0$
        \item $\forall x \in E$, $(N(x) = 0 \Rightarrow x = 0)$
        \item $\forall x, y \in E$, $N(x + y) \leq N(x) + N(y)$ (inégalité
            triangulaire)
        \item $\forall \lambda \in \mathbb{K}$, $\forall x \in E$, $N(\lambda x)
            \leq |\lambda| N(x)$
    \end{enumerate}
    Cette application est souvent dénotée par \GSnormeDef{.}{E} ou tout simplement
    \GSnorme{.}
\end{definition}

\begin{exercice}
    Montrer qu'on a
    $\forall \lambda \in \mathbb{K}$, $\forall x \in E$, $N(\lambda x) =
    |\lambda| N(x)$
\end{exercice}

En particulier, on a que E est un espace métrique ($d(x, y)$ = \GSnorme{x - y}).

La notion d'équivalence de norme est très intéressante, comme on le verra par
après. Elle permet de classer les espaces vectoriels normés.

\begin{definition}
    Soient deux normes \GSnormeDef{.}{1} \GSnormeDef{.}{2} sur E. On dit
	qu'elles sont équivalentes si $\exists k_{1}, k_{2} \in \mathbb{K}$ $\forall
	x \in E$ tel que $k_{1}$\GSnormeDef{x}{1}$\leq$\GSnormeDef{x}{2}$\leq$
	$k_{2}$\GSnormeDef{x}{1}
\end{definition}

Ce cours se base principalement sur les espaces de Banach, qui est une classe
intéressante d'espace vectoriel normé. Lorsque ce ne sera pas spécifié, l'espace
vectoriel étudié sera de Banach. Donnons maintenant la définition :

\begin{definition}
	Soit \GSnormedSpace{E}{\GSnorme{.}}. E est un espace de Banach si E est
	complet pour \GSnorme{.}.
\end{definition}

\begin{exercice}
	Soit E un $\mathbb{K}$ espace vectoriel, N une norme quelconque sur E, et
	$\mathbb{K}$ complet. 
	Montrez que E est un espace de Banach pour N.
	
	Pour cela, vous pouvez d'abord montrer que la convergence sur E se fait
	composante par composante. Il restera à déduire sa complétude suivant
	l'hypothèse sur $\mathbb{K}$.
\end{exercice}

\begin{exemple}
	Par l'exercice précédent, on en déduit que $M_{n}(\mathbb{K})$, l'ensemble
	des matrices carrées de dimension n à cofficients dans un corps complet
	$\mathbb{K}$ est un espace de Banach. En effet, celui-ci est de dimension
	finie $n^{2}$.
\end{exemple}

\begin{proposition}
	Soit $E = $\GSprodSet{i}{1}{n}{E} où \GSnormedSpace{$E_{i}$}{\GSnormeDef{.}{i}}
	est un espace de Banach pour $1 \leq i \leq n$.
	On note pour $x \in E$, $x = (x_{1}, \ldots, x_{n})$ où $x_{i} \in
	E_{i}$ pour $1 \leq i \leq n$.
	
	Alors, on peut définir les normes suivantes sur E :
	\begin{enumerate}
		\item \GSnormeDef{x}{1} = \GSsum{i}{1}{n}{\GSnormeDef{x_{i}}{i}}.
		\item \GSnormeDef{x}{\infty} = $\underset{1 \leq i \leq
			n}{max}$\GSnormeDef{x_{i}}{i}
	\end{enumerate}

	De plus, E est un espace de Banach pour ces deux normes.
\end{proposition}

\begin{exemple}
	On peut faire le lien avec la norme 1 et la norme infinie sur
	$\mathbb{R}^{n}$. On en déduit alors que $\mathbb{R}^{n}$ est un espace de
	Banach, qui avait été montré précédemment.
\end{exemple}

\begin{question}
	\begin{enumerate}
		\item Peut-on construire une norme sur tout espace vectoriel ?
			Sinon, quelle(s) condition(s) supplémentaire(s) a-t-on besoin ?
		\item Est-ce qu'un produit quelconque d'espaces de Banach est un
			espace de Banach ? (En supposant qu'on peut constuire une norme
			sur le produit.)
	\end{enumerate}
\end{question}

\section{Topologie des normes}

Sur un espace vectoriel normé donné, nous pouvons aussi construire une
topologie. Cela permettra de définir la continuité.

\begin{definition}
	Soit une norme \GSnorme{.} sur E. La topologie sur E associée à \GSnorme{.}
	est définie par l'ensemble des unions quelconques et des intersections
	finies de boules ouvertes associées à la norme \GSnorme{.}.
\end{definition}

\begin{proposition}
	Deux normes sont équivalentes si et seulement si elles définissent la même topologie.
\end{proposition}

Regardons maintenant deux théorèmes très importants. Nous supposons E un espace
vectoriel normé.

\begin{theorem}
	E est de dimension finie si et seulement si toutes les normes sont équivalentes.
\end{theorem}

\begin{theorem}
	\label{Riesz}
	E est de dimension finie si et seulement si sa boule unité fermée est compacte.
\end{theorem}

On en déduit alors une propriété assez intéressante :

\begin{proposition}
	Toutes les normes sur un espace vectoriel de dimension finie définissent la
	même topologie. Nous parlerons alors de \textbf{la} topologie de \textbf{la} norme de E.
\end{proposition}

\section{Application linéaire}
Nous allons maintenant étudier les applications linéaires. Nous savons déjà que
celles-ci représentent les morphismes entre espaces vectoriels, c'est-à-dire
qu'elles permettent de relier les structures entre deux espaces vectoriels.

Nous avons vu précédemment que lorsqu'on est sur un espace vectoriel normé, on
peut définir une norme, et que celle-ci définissait une topologie. Nous pouvons
alors parlé de continuité.

\begin{definition}
	L'ensemble des applications linéaire forme un espace vectoriel pour la
	multiplication scalaire habituelle et l'addition de fonction habituelle.
	Nous notons \GShomomorphisme{E}{F} l'espace vectoriel des applications
	linéaire de E dans F. (Si E = F, \GShomomorphisme{E}{F} =
	\GSendomorphism{E})
\end{definition}

Prenons une fonction $f \in$ \GShomomorphisme{E}{F} et $a \in E$.

\begin{proposition}
	Les assertions suivantes sont équivalentes :

	\begin{enumerate}
		\item f est continue.
		\item f est continue en a.
		\item f est continue en 0.
		\item f est bornée sur la boule unité de E.
		\item f est bornée sur tout ensemble bornée de E.
		\item $\exists k > 0, \forall x \in E,$ \GSnormeDef{f(x)}{F} $\leq$ k
			\GSnormeDef{x}{E}.
		\item f est lipschitzienne.
		\item f est uniformément continue.
	\end{enumerate}
\end{proposition}

On sait que la somme, le produit, la différence, ainsi que la multiplication par
un scalaire d'une fonction continue reste continue. On a alors que l'ensemble
des applications linéaire continues forment un espace vectoriel, dénoté
\GScontinueHomo{E}{F}. (Si E = F, \GScontinueHomo{E}{F} =
\GScontinueEndo{E})

En particulier, \GScontinueHomo{E}{F} (resp \GScontinueEndo{E}) est un sous
espaces vectoriel de \GShomomorphisme{E}{F} (resp \GSendomorphism{E}).

\begin{remarque}
	Pour une fonction quelconque entre deux espace métriques, nous avons
	toujours que la continuité uniforme implique la continuité, mais pas
	nécéssairement l'inverse (voir \GSfunction{f}{$\mathbb{R}$}{$\mathbb{R}$} : x $\rightarrow$ $x^2$ qui est continue mais pas uniformément).
\end{remarque}

Le théorème suivant nous donne une condition suffisante pour qu'une fonction
continue soit uniformément continue.

\begin{theorem} [Heine]
	Soit K un compact de E à valeur dans F, où E et F sont des espaces
	métriques. Soit \GSfunction{f}{K}{F} continue. Alors f est uniformément
	continue.
	\label{Heine}
\end{theorem}

\begin{exemple}
	Tout fonction f d'un intervalle \GSintervalCC{a}{b} dans $\mathbb{R}$, continue, est
	uniformément continue, comme
	\GSfunction{f}{\GSintervalCC{-1}{1}}{$\mathbb{R}$} : x $\rightarrow$ $x^2$.
\end{exemple}

Encore une fois, nous allons montrer une résultat important quand E est de
dimension finie.

\begin{proposition}
	Soit E de dimension finie, et F de dimension quelconque.
	
	Alors \GShomomorphisme{E}{F} = \GScontinueHomo{E}{F}, c'est-à-dire toute
	application linéaire est continue lorsque l'espace de départ est de
	dimension finie.
\end{proposition}

\begin{proof}
	Soit E de dimension n.
	On a que E est isomorphe à $\mathbb{K}^{n}$. Prenons $(e_{1}, e_{2}, \ldots,
	e_{n})$ une base de E.
	Soit $f \in $\GShomomorphisme{E}{F}. Montrons que $\forall x \in E$,
	\GSnormeDef{f(x)}{F}$ \leq $k\GSnormeDef{x}{E}.

	On a $x = $\GSsum{i}{1}{n}{$x_{i}e_{i}$}.
	On a $f(x) = f($\GSsum{i}{1}{n}{$x_{i}e_{i}$})
\end{proof}
