\chapter{Formules de Taylor}

\section{Formules de Taylor de type global}

\subsection{Formules de Taylor avec reste intégral}

Commençons d'abord par un lemme:

\begin{lemma}
	\label{lemma:taylor_formula_reste_derivative}
	Soient $I$ un intervalle de $\real$, $F$ un espace de Banach et
	$\GSfunction{g}{I}{F}$ une fonction $n+1$ fois dérivable.

	Alors:

	\begin{equation*}
		\displaystyle \dv{t} [g(t) + \sum_{p = 1}^{n} \frac{(1 - t)^{p}}{p!}
		g^{(p)}(t)] = \frac{(1 - t)^{n}}{t!} g^{(n+1)}(t)
	\end{equation*}
\end{lemma}

\ifdefined\outputproof
\begin{proof}

\end{proof}
\fi

Rappelons que l'intégrale de Riemann telle que construite pour une fonction d'un
intervalle $I$ réel à valeurs dans $\real$  peut-être construite de la même
manière pour une fonction d'un intervalle $I$ réel à valeur dans \textit{un
espace de Banach}. La construction est la même, et les résultats obtenus dans le
cas réel, peuvent aussi être démontrés dans le cas d'un espace de Banach
quelconque.

Nous obtenons alors une version intégrale du lemme
\ref{lemma:taylor_formula_reste_derivative}:

\begin{corollary}
	\label{corollary:taylor_formula_reste_integral}
	Supposons que $[0, 1] \subseteq I$, et $g$ de classe $\mathcal{C}^{n + 1}$.
	Alors on obtient:

	\begin{equation*}
		g(1) - g(0) - \displaystyle \sum_{p = 1}^{n} \frac{1}{p!} g^{(p)}(0) =
		\int_{0}^{1} \frac{(1 - t)^{n}}{n!} g^{(n + 1)}(t) dt
	\end{equation*}
\end{corollary}

\ifdefined\outputproof
\begin{proof}

\end{proof}
\fi

Posons une notation qui simplifiera l'écriture pour les prochains théorèmes.

\begin{notation}
	Soit $E$ un espace vectoriel normé. Soient $h \in E$ et $n \in \naturel$.

	On note $h^{n} = \underbrace{(h, \cdots, h)}_{\text{$n$-fois}} \in E^{n}$.
	On note aussi parfois $h^{[n]}$ ou $(h)^{n}$.
\end{notation}

Nous en venons alors au théorème de formule de Taylor avec reste intégral.

\begin{theorem}
	\label{theorem:taylor_formula_reste_integral}
	Soient $E$ et $F$ deux espaces de Banach. Soit $\mathcal{U}$ un ouvert de
	$E$.
	Soit $\GSfunction{f}{\mathcal{U}}{F}$ tel que $f \in \mathcal{C}^{n + 1}$.

	Soit $(x, h) \in (\mathcal{U} \cartesian E)$ tel que $\segment{x}{x + h}
	\subseteq \mathcal{U}$.

	Alors:
	\begin{align*}
		f(x + h) = f(x) & + \displaystyle \sum_{p = 1}^{n} \frac{1}{p!}
		\overbrace{d^{p}f(x)}^{\in \GScontinueHomo{E^{p}}{F}}
		\overbrace{(h^{p})}^{\in E^{p}} \\
		& + \int_{0}^{1} \frac{(1 - t)^{n}}{n!} \underbrace{[d^{n + 1}f(x +
		th)]}_{\in \GScontinueHomo{E^{n+1}}{F}} \underbrace{(h^{n+1})}_{\in E^{n +
		1}}
	\end{align*}

	Remarquons que $x + th \in \mathcal{U}$ par hypothèse sur $\mathcal{U}$,
	donc c'est bien défini.
\end{theorem}

\ifdefined\outputproof
\begin{proof}

\end{proof}
\fi

\subsection{Formules de Taylor-Lagrange}

Donnons un autre corollaire du lemme
\ref{lemma:taylor_formula_reste_derivative}.

\begin{corollary}
	Supposons que $g$ soit $n + 1$ fois dérivable.
	On suppose également qu'il existe $M \geq 0$ tel que $\forall t \in [0, 1]$,
	$\GSnormeDef{g^{(n + 1)}(t)}{F} \leq M$

	Alors:

	\begin{equation*}
		\GSnormeDef{g(1) - g(0) - \displaystyle \sum_{p = 1}^{n} \frac{1}{p!}
		g^{(p)}(0)}{F} \leq \frac{M}{(n + 1)!}
	\end{equation*}
\end{corollary}

\ifdefined\outputproof
\begin{proof}

\end{proof}
\fi

On en vient alors à la formule de Taylor-Lagrange:

\begin{theorem} [Formule de Taylor-Lagrange]
	\label{theorem:taylor_lagrange_formula}
	Soient $E$ et $F$ deux espaces de Banach. Soit $\mathcal{U}$ un ouvert de
	$E$.

	Soit $\GSfunction{f}{\mathcal{U}}{F}$ $n + 1$-fois différentiable.
	Soit $(x, h) \in (\mathcal{U} \cartesian E)$ tel que $\segment{x}{x + h}
	\subseteq \mathcal{U}$.

	On suppose que:
	\begin{equation*}
		\forall y \in \segment{x}{x + h}, \GSnormeDef{d^{n +
		1}f(y)}{\GScontinueHomo{E^{n + 1}}{F}} \leq M
	\end{equation*}
	avec $M \geq 0$.

	Alors:

	\begin{equation*}
		\GSnormeDef{f(x + h) - f(x) - \displaystyle \sum_{p = 1}^{n}
		\frac{1}{p!} [d^{p}f(x)](h^{p})}{F} \leq \frac{M}{(n + 1)!}
		\GSnormeDef{h}{E}^{n + 1}
	\end{equation*}
\end{theorem}

\ifdefined\outputproof
\begin{proof}

\end{proof}
\fi

\begin{exercice}
	-- Retrouver l'inégalité des accroissements finis en utilisant les formules
	de Taylor avec reste intégrale à l'ordre $1$ pour $f \in \mathcal{C}^{1}$.

	-- Montrer la formule de Taylor-Lagrange pour $f$ de classe $\mathcal{C}^{n
	+ 1}$ en utilisant la formule de Taylor avec reste intégrale.
\end{exercice}

\section{Formules de Taylor-Young}

\begin{theorem}
	\label{theorem:taylor_young_formula}
	Soient $E$ et $F$ des espaces vectoriels normés. Soit $\mathcal{U}$ un
	ouvert de $E$. Soit $a \in E$.

	Soit $\GSfunction{f}{\mathcal{U}}{F}$ $n$-fois différentiable.

	Alors:
	\begin{align*}
		f(a + h) = f(a) & + \displaystyle \sum_{p = 1}^{n} \frac{1}{p!}
		[d^{p}f(a)] (h^{p}) \\
						& + o(\GSnormeDef{h}{E}^{n})
	\end{align*}
\end{theorem}

\ifdefined\outputproof
\begin{proof}

\end{proof}
\fi
