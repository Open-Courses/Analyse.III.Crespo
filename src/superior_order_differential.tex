\chapter{Différentielles d'ordre supérieur}

\section{Différentielle seconde}

\subsection*{Notations et objectifs}
Nous travaillerons pour l'instant avec $E$ et $F$ des espaces de Banach.
$\mathcal{U}$ sera un ouvert de $E$.

On prendra aussi une fonction $\GSfunction{f}{\mathcal{U}}{F}$ différentiable, et
sa différentielle sera notée
$\GSfunction{df}{\mathcal{U}}{\GScontinueHomo{E}{F}} : x \rightarrow df(x)$.

On va s'intéresser à la différentielle de $df$.

\subsection{Définition et propriétés}

\begin{definition}
	Soit $a \in \mathcal{U}$. On dit que $f$ est \textbf{deux fois
	différentiable en a} s'il existe un voisinage $\mathcal{V}_{a} \subseteq
	\mathcal{U}$ de $a$ ouvert tel que:

	-- $f$ est différentiable sur $\mathcal{V}_{a}$.

	-- $\GSfunction{df}{\mathcal{V}_{a}}{\GScontinueHomo{E}{F}}$ est diffentiable
	en $a$.
	Dans ce cas, on note $d^{2}f(a)$ la différentielle de $df$ en $a$, et on dit
	que c'est \textbf{la différentielle seconde de $f$ en $a$}.
\end{definition}

Dans la définition, nous demandons que $f$ soit différentiable sur un
voisinage ouvert de $a$ car nous voulons définir $df$ sur un ouvert. $f$ doit
donc être différentiable sur un ouvert pour que $df$ soit définie sur un ouvert.

Remarquons également que $d^{2}f(a) \in
\GScontinueHomo{E}{\GScontinueHomo{E}{F}}$.

\begin{definition}
	On dit que \textbf{$f$ est deux fois différentiable} si $f$ est deux fois
	différentiable en tout point de $\mathcal{U}$.

	On a alors:

	- $\GSfunction{f}{\mathcal{U}}{F}$.

	- $\GSfunction{df}{\mathcal{U}}{\GScontinueHomo{E}{F}}$.

	- $\GSfunction{d^{2}f}{\mathcal{U}}{\GScontinueHomo{E}}{\GScontinueHomo{E}{F}}$.
\end{definition}

Comme pour les fonctions dérivables, on définit les classes $\mathcal{C}^{n}$.

\begin{definition}
	On dit que $f \in \mathcal{C}^{2}$ si $f$ est deux fois différentiable et
	que sa différentielle seconde $d^{2}f$ est continue.
\end{definition}

La particularité de $d^{2}f(a)$ est d'appartenir à
$\GScontinueHomo{E}{\GScontinueHomo{E}{F}}$ qui n'a pas l'air d'être un espace
très facile à manipuler. Cependant, nous pouvons utiliser une isométrie vers un
espace plus facile à manipuler: celui des fonctions bilinéaire de $E \cartesian
E$ dans $F$.

\begin{proposition}
	Soit $E, F, G$ des espaces vectoriels normés.

	Alors $\GScontinueHomo{E}{\GScontinueHomo{F}{G}} \identification
	\GScontinueHomo{E,F}{G}$.
\end{proposition}

\ifdefined\outputproof
\begin{proof}

\end{proof}
\fi

On se retrouve donc avec une identification de $d^{2}f(a)$ en tant
qu'application bilinéaire de telle manière que pour tout $h \in E, k \in F$, on
a $[d^{2}f(a)(h)](k) \identification d^{2}f(a)(h, k)$.

Dans le premier membre de l'isométrie, on a $d^{2}f(a)(h) \in
\GScontinueHomo{E}{F}$. Elle doit donc prendre une valeur dans $E$ pour être
évaluée dans $F$, et donc quand on applique en l'élément $k$, on obtient
$[d^{2}f(a)(h)](k) \in F$.

L'autre membre prend un élément de $(h, k) \in E \cartesian F$.

Dans la suite, nous considérerons la différentielle seconde grace à son
isométrie.

En réalité, la différentielle seconde possède une propriété intéressante: elle
est symétrique.

\begin{theorem}
	[symétrie de Schwarz \footnote{Karl Hermann Amandus Schwarz (25/01/1843 -
	30/11/1921) mathématicien allemand. A ne pas confondre avec Laurant
	Schwarz\textbf{t}, mathématicien français mort en 2004}]
	\label{theorem_schwarz}

	Si $f$ est deux fois différentiable en $a$, alors $d^{2}f(a)(h, k) =
	d^{2}f(a)(k, h)$ pour tout $(h, k) \in E \cartesian F$, c'est-à-dire que
	$d^{2}f(a)$ bilinéaire continue symétrique.
\end{theorem}

Remarquons d'abord que ce théorème n'est pas si évident à voir si nous ne
faisons pas l'identification à travers l'isométrie. En effet, la symétrie dirait
que $[d^{2}f(a)(h)](k) = [d^{2}f(a)(k)](h)$

\ifdefined\outputproof
\begin{proof}

\end{proof}
\fi

\subsection{Dérivées partielles secondes}

Soient $(E_{i})_{1 \leq i \leq n}$ des espaces vectoriels normées, $E =
\GSprodSet{i}{1}{n}{E}$, et $F$ un espace vectoriel normé.

Soient, pour chaque $i$, $\mathcal{U}_{i}$ un ouvert de $E_{i}$, et posons
$\mathcal{U} = \GSprodSet{i}{1}{n}{\mathcal{U}}$.

Soit $\GSfunction{\mathcal{U}}{F}$ une application.

\begin{proposition}
	Supposons que $f$ soit deux fois différentiable en un point $a \in
	\mathcal{U}$. Par définition, il existe un voisinage ouvert
	$\mathcal{V}_{a}$, compris dans $\mathcal{U}$ tel que $f$ est
	différentiable sur $\mathcal{V}_{a}$.
	Nous savons que $f$ admet des dérivées partielles en tout point de $\mathcal{V}_{a}$.

	Alors, pour tout $1 \leq j \leq n$, les applications:

	\begin{equation}
		\pdv{f}{x_{j}} : \mathcal{V}_{a} \rightarrow \GScontinueHomo{E_{j}}{F}
	\end{equation}

	admettent des dérivées partielles selon $E_{i}$ pour tout $1 \leq i \leq n$,
	en tout point $x \in \mathcal{V}_{a}$, et on note:

	\begin{equation}
		\pdv{f}{x_{i}}{x_{j}} (x) := \pdv{x_{i}}(\pdv{f}{x_{j}}) (x)
	\end{equation}

	$\pdv{f}{x_{i}}{x_{j}} \in
	\GScontinueHomo{E_{i}}{\GScontinueHomo{E_{j}}{F}} \identification
	\GScontinueHomo{E_{i}, E_{j}}{F}$.

	De plus, $d^{2}f(a)$ peut être calculée grace à ces dernières avec la
	formule:

	\begin{equation}
		d^{2}f(a)(h, k) = \displaystyle \sum_{1 \leq i, j \leq n}
		\pdv{f}{x_{i}}{x_{j}} (a) (h_{i}, k_{j})
	\end{equation}

	avec $h = (h_{1}, \cdots, h_{n})$, et $k = (k_{1}, \cdots, k_{n})$ dans $E$.

\end{proposition}

On appelle $\pdv{f}{x_{i}}{x_{j}} (a)$, pour $1 \leq i, j \leq n$, \textbf{les
dérivées partielles secondes de $f$ en $a$}.

De plus, on note $\pdv[2]{f}{x_{i}} (a) := \pdv{f}{x_{i}}{x_{i}} (a)$ pour tout $1 \leq i
\leq n$.

\ifdefined\outputproof
\begin{proof}
   % Remarquons que si $f$ est deux fois différentiable en $a$, on a que $f$ est différentiable sur un
	%$\mathcal{V} \subseteq \mathcal{U}$, et même que $f$ est de classe
	%$\mathcal{C}^{1}$.
	%Donc, $\GSfunction{df}{\mathcal{V}}{\GScontinueHomo{E}{F}}$ admet des
	%dérivées partielles selon $E_{i}$ pour chaque $1 \leq i \leq n$ au point $a$.

	%On a donc que les applications:

	%\begin{equation}
		%\pdv{x_{i}} (df) (a) \in \GScontinueHomo{E_{i}}{\GScontinueHomo{E}{F}}
	%\end{equation}

	%existent. N'oublions pas que $\GScontinueHomo{E_{i}}{\GScontinueHomo{E}{F}}
	%\identification \GScontinueHomo{E_{i}, E}{F}$.
\end{proof}
\fi

Nous obtenons alors, pour le théorème de symétrie de Schwarz:

\begin{theorem}
	Si $f$ est deux fois différentiable au point $a$, alors on a, pour tout $h,
	k \in E$, et pour tout $1 \leq i, j \leq n$:

	\begin{equation}
		\overbrace{\pdv{f}{x_{i}}{x_{j}} (a)}^{\in \GScontinueHomo{E_{i},
		E_{j}}{F}} (h_{i}, k_{j}) = \underbrace{\pdv{f}{x_{j}}{x_{i}} (a)}_{\in
			\GScontinueHomo{E_{j}, E_{i}}{F}}
		(k_{j}, h_{i})
	\end{equation}
\end{theorem}

\ifdefined\outputproof
\begin{proof}

\end{proof}
\fi

% Voir les remarques page 6 Aline.

\begin{proposition}
	Pour que $f$ soit de classe $\mathcal{C}^{2}$ sur $\mathcal{U}$, il est nécessaire et
	suffisant que $f$ admettent des dérivées partielles secondes en tout point
	de $\mathcal{U}$ et que les applications:

	\begin{equation}
		\pdv{f}{x_{i}}{x_{j}} (a) : \mathcal{U} \rightarrow
		\GScontinueHomo{E_{i}, E_{j}}{F}
	\end{equation}
	soient continues.
\end{proposition}

\subsection{Dérivées directionnelles du second ordre}

Soient $E, F$ des espaces vectoriels normés, et $\mathcal{U}$ un ouvert de $E$.

Supposons que $\GSfunction{f}{\mathcal{U}}{F}$ soit deux fois différentiable au
point $a \in \mathcal{U}$.
On sait donc qu'en particulier, il existe un voisinage ouvert $\mathcal{V}_{a}$ de $a$ tel
que $f$ est différentiable sur $\mathcal{V}_{a}$

On a alors vu au chapitre \ref{chap:differential} que $f$ admet des
dérivées directionnelles dans toutes les directions non nulles $k \in E$, et
$\partial_{k}{f} (x) \in F$ pour tout $x \in \mathcal{V}_{a}$.

On a donc:

\begin{equation}
	\partial_{k}f : \mathcal{V}_{a} \rightarrow F
\end{equation}

\begin{proposition}
	Soit $k \in E$ non nul.

	La fonction $\partial_{k}f : \mathcal{V}_{a} \rightarrow F$ admet des
	dérivées directionnels dans toutes les directions non nulles $h \in E$, et
	on note:

	\begin{equation}
		\partial_{h, k}^{2} f (a) := \partial_{h} (\partial_{k}(f)) (a)
	\end{equation}

	De plus, $\partial_{h, k}^{2} f (a) = d^{2}f(a) (h, k)$, et nous avons:

	\begin{enumerate}
		\item $\partial_{k_{1} + \lambda k_{2}, h}^{2} f(a) =
			\partial_{k_{1}, h}^{2}f(a) + \lambda \partial_{k_{2}, h}^{2} f(a)$
		\item $\partial_{k, h_{1} + \lambda h_{2}}^{2} f(a) = \partial_{k,
				h_{1}}^{2} f(a) + \lambda \partial_{k, h_{2}}^{2} f(a)$
		\item $\partial_{k, h}^{2} f(a) = \partial_{h, k}^{2} f(a)$
	\end{enumerate}
\end{proposition}

\ifdefined\outputproof
\begin{proof}

\end{proof}
\fi

\section{Différentielles d'ordre supérieur}

\subsection{Définitions et quelques propriétés}

\subsection{Règles de calcul et résultats généralisés}

Dans cette partie, nous allons étudier les propriétés de différentiabilité de
certains opérateurs, comme les opérateurs linéaires.

\subsubsection{Applications linéaires}

Prenons $T \in \GScontinueHomo{E}{F}$ où $E$ et $F$ sont des espaces vectoriels
normés.
On a vu que $T$ est différentiable et que $\forall x \in E$, $dT(x) = T$,
c'est-à-dire que la fonction différentielle qui a un $x \in E$, associée la
différentielle de $T$ en $x$, est constante.

On a alors que la différentielle seconde de $T$ est constante. En particulier,
$T \in \mathcal{C}^{\infty}(E, F)$, et $d^{n}T = 0$ pour $n \geq 2$.

\subsubsection{Applications bilinéaires}

Soient $E, F, G$ des espaces vectoriels normés.
Soit $T \in \GScontinueHomo{E, F}{G}$.

On sait que $T$ est de classe $\mathcal{C}^{1}$, de plus:
\begin{align*}
	dT : & E \cartesian F \rightarrow \GScontinueHomo{E, F}{G} \\
	& (x, y) \rightarrow dT(x, y)
\end{align*}

où on a:
\begin{align*}
	dT(x, y) : & E \cartesian F \rightarrow G \\
	& (a, b) \rightarrow T(a, y) + T(x, b)
\end{align*}

et $dT$ est linéaire et continue.

On a donc $dT$ qui est différentiable, et sa différentielle est constante.
Donc $d^{2}T$ est aussi différentiable, et $d^{3}T$ est nulle.

Nous avons donc $d^{n}T = 0$ pour $n \geq 3$. D'où $T \in
\mathcal{C}^{\infty}(E \cartesian F, G)$.

\subsubsection{Applications k-linéaires}

Nous pouvons alors généraliser aux applications $k$-linéaires.

\begin{proposition}
	Soient $E_{1}, \cdots, E_{k}, F$ des espaces vectoriels normés.

	Soit $T \in \GScontinueHomo{E_{1}, \cdots, E_{k}}{F}$ une application
	$k$-linéaire.

	Alors $T$ est de classe $\mathcal{C}^{\infty}$, et $d^{n}T = 0$ pour tout $n
	\geq k + 1$.
\end{proposition}

\begin{proof}

\end{proof}

\subsubsection{Règle de Leibniz}

\begin{proposition}
	Soient $E, F_{1}, F_{2}, G$ des espaces vectoriels normés.

	Soit $\mathcal{U} \subseteq E$ un ouvert de $E$.
	Soient $\GSfunction{f}{\mathcal{U}}{F_{1}}$, et
	$\GSfunction{g}{\mathcal{U}}{F_{2}}$ des applications $n$-fois différentiables
	(resp. de classe $\mathcal{C}^{n}$)

	Soit $b \in \GScontinueHomo{F_{1}, F_{2}}{G}$.
	Alors l'application:

	\begin{align*}
		W : & \mathcal{U} \rightarrow G \\
			& x \rightarrow b(f(x), g(x))
	\end{align*}

	est $n$-fois différentiables (resp. de classe $\mathcal{C}^{n}$).
\end{proposition}

\begin{proof}

\end{proof}

\subsubsection{Règle de différentiabilité en chaîne}

Nous avons vu la chain-rule (\ref{theorem:chain_rule}) qui nous dit que la
composition de deux fonctions différentiables est encore différentiable.

Nous allons généraliser ce théorème à des applications $n$-fois différentiables
(resp. de classe $\mathcal{C}^{n}$).

\begin{proposition}
	Soient $E, F, G$ des espaces vectoriels normés.

	Soient $\mathcal{U}$ un ouvert de $E$ et $\mathcal{V}$ un ouvert de $F$.

	Soient $\GSfunction{f}{\mathcal{U}}{F}$, et $\GSfunction{g}{\mathcal{V}}{G}$
	tel que $f(\mathcal{U}) \subseteq \mathcal{V}$ et tel que $f$ et $g$ soient
	$n$-fois différentiables (resp. de classe $\mathcal{C}^{n}$).

	Alors $\GSfunction{g \circ f}{\mathcal{U}}{G}$ est $n$-fois différentiable
	(resp. de classe $\mathcal{C}^{n}$).
\end{proposition}

\begin{proof}

\end{proof}

\subsubsection{Différentielle composante par composante}

Soient $E, F_{1}, \cdots, F_{k}$ des espaces vectoriels normés.

\begin{proposition}
	Soit $\mathcal{U}$ un ouvert de $E$.
	Soit $\GSfunction{f}{\mathcal{U}}{\displaystyle \prod_{l = 1}^{k} F_{l}}$ une
	application.
	Alors, les assertions suivantes sont équivalentes:

	\begin{itemize}
		\item Les applications composantes sont $n$-fois différentiables (resp.
			de classe $\mathcal{C}^{n}$).
		\item $f$ est $n$-fois différentiable (resp. de classe
			$\mathcal{C}^{n}$).
	\end{itemize}

	De plus, $d^{n}f_{l}(x) = p_{F_{l}} \circ d^{n} f(x)$ pour tout $x \in
	\mathcal{U}$.
\end{proposition}

\begin{proof}

\end{proof}

\subsubsection{Applications inverses}

Soient $E$ et $F$ des espaces de Banach.

Nous avons construit l'application:

\begin{align*}
	\phi : & Isom(E, F) \rightarrow Ison(F, E)
	& u \rightarrow u^{-1}
\end{align*}

et avons montré qu'elle est de classe $\mathcal{C}^{1}$. Nous allons généraliser
ce résultat.

\begin{proposition}
	L'application inverse $\phi$ est de classe $\mathcal{C}^{\infty}$.
\end{proposition}

\begin{proof}

\end{proof}

\begin{proposition}
	Soient $E$, $F$ des espaces de Banach.
	Soient $\mathcal{U}$ un ouvert de $E$, et $\mathcal{V}$ un ouvert de $F$.

	Soit $\GSfunction{f}{\mathcal{U}}{\mathcal{V}}$ un $\mathcal{C}^{-1}$
	difféomorphisme.

	Si $f$ est de classe $\mathcal{C}^{n}$, $f^{-1}$ est de classe
	$\mathcal{C}^{n}$.
\end{proposition}

\begin{proof}

\end{proof}

\subsubsection{Théorème d'inversion local généralisé}

Il en résulte comme corollaire un théorème qui généralise le théorème
d'inversion local (\ref{theorem:local_inversion}).

\begin{theorem}
	Soient $E$, $F$ des espaces de Banach. Soit $\mathcal{U}$ un ouvert de $E$.
	Soit $\GSfunction{f}{\mathcal{U}}{F}$ une fonction de classe
	$\mathcal{C}^{n}$.

	Si $df(a) \in Isom(E, F)$ pour un $a \in \mathcal{U}$, alors il existe un
	voisinage ouvert $\mathcal{V}$ de $f(a)$ dans $F$, et $\mathcal{W} \subseteq
	\mathcal{U}$ tel que:

	\begin{itemize}
		\item $f_{\mathcal{V}}(\mathcal{W}) = \mathcal{V}$
		\item $f_{\mathcal{V}}^{\mathcal{W}} : \mathcal{W} \rightarrow
			\mathcal{V}$ est un $\mathcal{C}^{n}$ difféomorphisme.
	\end{itemize}
\end{theorem}

\begin{proof}

\end{proof}

\subsubsection{Théorème des fonctions implicites généralisé}

\begin{theorem}
	Soient $E$, $F$, $G$ des espaces de Banach.
	Soit $\mathcal{U}$ un ouvert de $E \cartesian F$.
	Soit $(a, b) \in \mathcal{U}$.

	Soit $\GSfunction{f}{\mathcal{U}}{G}$ une application de classe
	$\mathcal{C}^{n}$ tel que:

	\begin{itemize}
		\item $f(a, b) = 0$
		\item $\partial_{2}f (a, b) \in Isom(F, G)$
	\end{itemize}

	Alors il existe un voisinage ouvert $\mathcal{V}$ de $a$ dans $E$ et
	$\mathcal{W}$ voisinage ouvert de $b$ dans $F$ tel que $\mathcal{V}
	\cartesian \mathcal{W} \subseteq \mathcal{U}$ et il existe une fonction
	$\GSfunction{\phi}{\mathcal{V}}{\mathcal{W}}$ tel que

	\begin{align*}
	\left\{
		\begin{array}{r c l}
			(x, y) \in \mathcal{V} \cartesian \mathcal{W} \\
			f(x, y) = 0
		\end{array}
	\right.
	\Leftrightarrow
	\left\{
		\begin{array}{r c l}
			\phi(x) &=& y \\
			x \in \mathcal{V}
		\end{array}
	\right.
	\end{align*}
\end{theorem}

\begin{proof}

\end{proof}

\subsection{Dérivées partielles d'ordre supérieur}

\subsection{Exemple fondamental $E = \real^{d}$}

\subsection{Notations multi-indicielles}

