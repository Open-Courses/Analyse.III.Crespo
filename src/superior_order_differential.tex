\chapter{Différentielles d'ordre supérieur}

\section{Différentielle seconde}

\subsection*{Notations et objectifs}
Nous travaillerons pour l'instant avec $E$ et $F$ des espaces de Banach.
$\mathcal{U}$ sera un ouvert de $E$.

On prendra aussi une fonction $\GSfunction{f}{\mathcal{U}}{F}$ différentiable, et
sa différentielle sera notée
$\GSfunction{df}{\mathcal{U}}{\GScontinueHomo{E}{F}} : x \rightarrow df(x)$.

On va s'intéresser à la différentielle de $df$.

\subsection{Définition et propriétés}

\begin{definition}
	Soit $a \in \mathcal{U}$. On dit que $f$ est \textbf{deux fois
	différentiable en a} s'il existe un voisinage $\mathcal{V}_{a} \subseteq
	\mathcal{U}$ de $a$ ouvert tel que:

	-- $f$ est différentiable sur $\mathcal{V}_{a}$.

	-- $\GSfunction{df}{\mathcal{V}_{a}}{\GScontinueHomo{E}{F}}$ est diffentiable
	en $a$.
	Dans ce cas, on note $d^{2}f(a)$ la différentielle de $df$ en $a$, et on dit
	que c'est \textbf{différentielle seconde de $f$ en $a$}.
\end{definition}

Dans la définition, nous demandons que $f$ soit différentiable sur un
voisinage ouvert de $a$ car nous voulons définir $df$ sur un ouvert. $f$ doit
donc être différentiable sur un ouvert pour que $df$ soit définie sur un ouvert.

Remarquons également que $d^{2}f(a) \in
\GScontinueHomo{E}{\GScontinueHomo{E}{F}}$.

\begin{definition}
	On dit que \textbf{$f$ est deux fois différentiable} si $f$ est deux fois
	différentiable en tout point de $\mathcal{U}$.

	On a alors:

	- $\GSfunction{f}{\mathcal{U}}{F}$.

	- $\GSfunction{df}{\mathcal{U}}{\GScontinueHomo{E}{F}}$.

	- $\GSfunction{d^{2}f}{\mathcal{U}}{\GScontinueHomo{E}}{\GScontinueHomo{E}{F}}$.
\end{definition}

Comme pour les fonctions dérivables, on définit les classes $\mathcal{C}^{n}$.

\begin{definition}
	On dit que $f \in \mathcal{C}^{2}$ si $f$ est deux fois différentiable et
	que sa différentielle seconde $d^{2}f$ est continue.
\end{definition}

La particularité de $d^{2}f(a)$ est d'appartenir à
$\GScontinueHomo{E}{\GScontinueHomo{E}{F}}$ qui n'a pas l'air d'être un espace
très facile à manipuler. Cependant, nous pouvons utiliser une isométrie vers un
espace plus facile à manipuler: celui des fonctions bilinéaire de $E \cartesian
E$ dans $F$.

\begin{proposition}
	Soit $E, F, G$ des espaces vectoriels normés.

	Alors $\GScontinueHomo{E}{\GScontinueHomo{F}{G}} \isometric
	\GScontinueHomo{E \cartesian F}{G}$.
\end{proposition}

\begin{proof}
	
\end{proof}

On se retrouve donc avec une identification de $d^{2}f(a)$ en tant
qu'application bilinéaire de telle manière que pour tout $h \in E, k \in F$, on
a $[d^{2}f(a)(h)](k) \isometric d^{2}f(a)(h, k)$.

Dans le premier membre de l'isométrie, on a $d^{2}f(a)(h) \in
\GScontinueHomo{E}{F}$. Elle doit donc prendre une valeur dans $E$ pour être
évaluée dans $F$, et donc quand on applique en l'élément $k$, on obtient
$[d^{2}f(a)(h)](k) \in F$.

L'autre membre prend un élément de $(h, k) \in E \cartesian F$.

Dans la suite, nous considérerons la différentielle seconde grace à son
isométrie.

En réalité, la différentielle seconde possède une propriété intéressante: elle
est symétrique.

\begin{theorem}
	[symétrie de Schwarz \footnote{Karl Hermann Amandus Schwarz (25/01/1843 -
	30/11/1921) mathématicien allemand. A ne pas confondre avec Laurant
	Schwarz\textbf{t}, mathématicien français mort en 2004}]
	\label{theorem_schwarz}

	Si $f$ est deux fois différentiable en $a$, alors $d^{2}f(a)(h, k) =
	d^{2}f(a)(k, h)$ pour tout $(h, k) \in E \cartesian F$, c'est-à-dire que
	$d^{2}f(a)$ bilinéaire continue symétrique.
\end{theorem}

Remarquons d'abord que ce théorème n'est pas si évident à voir si nous ne
faisons pas l'identification à travers l'isométrie. En effet, la symétrie dirait
que $[d^{2}f(a)(h)](k) = [d^{2}f(a)(k)](h)$

\begin{proof}
	
\end{proof}
