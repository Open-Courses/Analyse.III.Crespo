\chapter*{Introduction}

Ce cours se concentrera sur l'étude des applications différentiables dans les
espaces de Banach construits sur le corps $\mathbb{R}$ ou le corps $\mathbb{C}$.
Certaines propriétés peuvent être généralisées à n'importe quel corps de base,
comme les notions de différentielles.

Dans le chapitre 1, nous commencerons d'abord par rappeler quelques notions sur
les normes, les applications linéaires et multilinéaires.

Dans le chapitre 2, nous regarderons les notions des différentiabilités, de propriétés s'y
rapportant.

Nous pourrons alors étudier des théorèmes comme le théorème d'inversion locale
et globale.

Les notions d'espaces vectoriels, de dimension, de corps et d'applications
linéaires sont supposées connues.

Les espaces vectoriels seront dénotés par les lettres E, F, G, H, \ldots, les
applications linéaires par f, g, h, \ldots et le corps de base par $\mathbb{K}$.
