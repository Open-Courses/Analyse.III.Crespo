\chapter{Théorème d'inversion locale}

\section{Homéomorphismes et difféomorphismes}

\begin{definition}
\label{definition_homeomorphisme}
	Une fonction $\GSfunction{f}{X}{Y}$ où $X$ et $Y$ sont deux espaces
	topologiques est un homémorphisme si elle est bijective, continue, et à
	réciproque continue ($f^{-1}$ continue).
\end{definition}

\begin{definition}
\label{definition_diffeomorphism}
Une fonction $\GSfunction{f}{\mathcal{U}}{\mathcal{V}}$ où $\mathcal{U}
\subseteq E$ et $\mathcal{V} \subseteq F$ sont des ouverts et $E$, $F$ des espaces
	vectoriels normés est un difféomorphisme si elle est bijective,
	de classe $\mathcal{C}^{1}$ et sa réciproque est de classe
	$\mathcal{C}^{-1}$.
\end{definition}

\begin{proposition}
	Soit $\GSfunction{f}{\mathcal{U}}{\mathcal{V}} \mathcal{U} \subseteq
	E$ et $\mathcal{V} \subseteq F$ de classe $\mathcal{C}^{1}$, bijective tel
	que $f^{-1}$ continue en $f(a) = b$.

	On a que $f^{-1}$ est différentiable en $b$ ssi $df(a) \in Isom(E, F)$. On a
	alors $df^{-1}(b) = {df(a)}^{-1}$.
\end{proposition}

\begin{proof}
	
\end{proof}

\begin{proposition}
	Soit $\GSfunction{f}{\mathcal{U}}{\mathcal{V}} \mathcal{U} \subseteq
	E$ et $\mathcal{V} \subseteq F$ un homéomorphisme. Alors on a que $f$ est un
	difféomorphisme ssi $\forall x \in E$, $df(x) \in Isom(E, F)$.
\end{proposition}

\begin{proof}
	
\end{proof}

\section{Théorèmes d'inversion locale}
	Voir
	\href{http://fr.wikipedia.org/wiki/Th%C3%A9or%C3%A8me_d%27inversion_locale}{wikipedia}
\subsection{Introduction et motivation}
