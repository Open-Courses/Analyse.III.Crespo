\chapter{Théorème d'inversion locale}

\section{Homéomorphismes et difféomorphismes}

\begin{definition}
	\label{definition:homeomorphisme}
	Soient $(X, \tau_{X})$ et $(Y, \tau_{Y})$ deux espaces topologiques.

	Une fonction $\GSfunction{f}{X}{Y}$ est \textbf{un homémorphisme} si elle
	est bijective, continue, et à réciproque continue ($f^{-1}$ continue).
\end{definition}

% TODO: Remarques page 26.

\begin{definition}
	Soient $(X, \tau_{X})$ et $(Y, \tau_{Y})$ deux espaces topologiques.

	Une fonction $\GSfunction{f}{X}{Y}$ est \textbf{ouverte} (resp.
	\textbf{fermée}) si pour tout ouvert $\mathcal{O}_{X}$ de $(X, \tau_{X})$,
	$f(\mathcal{O}_{X})$ est un ouvert de $(Y, \tau_{Y})$ (resp. pour tout fermé
	$\mathcal{F}_{X}$ de $(X, \tau_{X})$, $f(\mathcal{F}_{X})$ est un fermé de
	$(Y, \tau_{Y})$).
\end{definition}

\begin{theorem}
	Soient $(X, \tau_{X})$ et $(Y, \tau_{Y})$ deux espaces topologiques.

	Soit $\GSfunction{f}{X}{Y}$ une fonction bijective continue. Alors, les
	assertions suivantes sont équivalentes:

	\begin{itemize}
		\item $f$ est un homéomorphisme.
		\item $f$ est ouverte.
		\item $f$ est fermée.
	\end{itemize}
\end{theorem}

\ifdefined\outputproof
\begin{proof}

\end{proof}
\fi

\begin{definition}
	\label{definition:diffeomorphism}
	Soient $E$, $F$ des espaces vectoriels normés.

	Soient $\mathcal{U}$ un ouvert de $E$ et $\mathcal{V}$ un ouvert de $F$.

	Soit $\GSfunction{f}{\mathcal{U}}{\mathcal{V}}$.

	On dit que $f$ est \textbf{un difféomorphisme} si elle est bijective,
	de classe $\mathcal{C}^{1}$ et sa réciproque est de classe
	$\mathcal{C}^{-1}$.
\end{definition}

\begin{remarque}
	\begin{itemize}
		\item Un $\mathcal{C}^{1}$-difféomorphisme est un homémorphisme.
		\item En revanche, un homéomorphisme de classe $\mathcal{C}^{1}$ n'est
			pas nécessairement un $\mathcal{C}^{1}$-difféomorphisme.
			% TODO: Donner l'exemple page 26 Sam
	\end{itemize}
\end{remarque}

\begin{proposition}
	Soient $E, F$ deux espaces vectoriels normés.

	Soient $\mathcal{U}$ un ouvert de $E$, $\mathcal{V}$ un ouvert de $F$, et
	$a \in \mathcal{U}$.

	Soit $\GSfunction{f}{\mathcal{U}}{\mathcal{V}}$ une application bijective et
	différentiable en $a$.

	Alors, les assertions suivantes sont équivalentes:

	\begin{itemize}
		\item $f^{-1}$ est différentiable en $b := f(a)$
		\item $df(a) \in \GSisomorphismeHomo{E}{F}$
	\end{itemize}

	On a alors l'inverse de $df(a)$ qui est donné par $[{df(a)}]^{-1} =
	d(f^{-1})(b)$.
\end{proposition}

\ifdefined\outputproof
\begin{proof}

\end{proof}
\fi

\begin{proposition}
	\label{prop:homeo_diffeo_ssi_dfx_isom}
	Soient $E, F$ deux espaces vectoriels normés.

	Soient $\mathcal{U}$ un ouvert de $E$ et $\mathcal{V}$ un ouvert de $F$.

	Soit $\GSfunction{f}{\mathcal{U}}{\mathcal{V}}$ un homéomorphisme de classe
	$\mathcal{C}^{1}$.

	Alors les assertions suivantes équivalentes:

	\begin{itemize}
		\item $f$ est un $\mathcal{C}^{1}$-difféomorphisme
		\item $df(a) \in \GSisomorphismeHomo{E}{F}$ pour tout $a \in \mathcal{U}$
	\end{itemize}
\end{proposition}

\ifdefined\outputproof
\begin{proof}

\end{proof}
\fi

\section{Exemple fondamental}

Prenons $E = F = \real^{n}$ pour $n \in \naturel^{> 0}$.

Soit $\mathcal{U}$ un ouvert de $\real^{n}$.
Soit $\GSfunction{f}{\mathcal{U}}{\real^{n}}$.

Supposons que $f$ soit différentiable sur $\mathcal{U}$, et notions $f_{i}$ ses
applications composantes. Alors les dérivées partielles de $f$ existent pour
tout $j$ entre $1$ et $n$.

Notons $\jacobienneMatrix{f}{a}$ sa matrice jacobienne en $a$.

Supposons que $\mathcal{V} = f(\mathcal{U})$ et que $f$ soit un homéomorphisme
de $\mathcal{U}$ dans $\mathcal{V}$.

Alors, pour que $f$ soit un difféomorphisme de classe $\mathcal{C}^{1}$, il faut
et il suffit, par \ref{prop:homeo_diffeo_ssi_dfx_isom}, que pour tout $a \in
\mathcal{U}$, $df(a)$ soit un isomorphisme de $\real^{n}$ dans $\real^{n}$,
c'est-à-dire, par la correspondance matricielle, que $\jacobienneMatrix{f}{a}$
soit inversible (ie $det(\jacobienneMatrix{f}{a}) \neq 0$) pour tout $a \in
\mathcal{U}$.

\section{Théorème d'inversion locale}

\begin{theorem} [d'inversion locale]
	\label{theorem:local_inversion}
	Soient $E, F$ des espaces de Banach.
	Soit $\mathcal{U}$ un ouvert de $E$.

	Soit $\GSfunction{f}{\mathcal{U}}{F}$ une application différentiable sur
	$\mathcal{U}$.

	Soit $a \in \mathcal{U}$ tel que $df(a) \in \GSisomorphismeHomo{E}{F}$.

	Alors il existe un voisinage $\mathcal{V}_{a}$ de $a$ compris dans
	$\mathcal{U}$, et un voisinage $\mathcal{W}_{b}$ de $b := f(a)$ tel que

	\begin{itemize}
		\item $f^{|\mathcal{W}_{b}}_{|\mathcal{V}_{a}}$ est un $\mathcal{C}^{1}$-difféormorphisme.
		\item $f(\mathcal{V}_{a}) = \mathcal{W}_{b}$.
	\end{itemize}
\end{theorem}

\ifdefined\outputproof
\begin{proof}

\end{proof}
\fi

Rappelons la définition d'\textit{une fonction (strictement) contractante}.

\begin{definition}
	Soient $(X, d_{X})$, $(Y, d_{Y})$ deux espaces métriques.

	Soit $\GSfunction{f}{(X, d_{X})}{(Y, d_{Y})}$ une fonction.

	On dit que $f$ est \textbf{contractante} s'il existe $k \in [0, 1]$ tel que

	\begin{equation*}
		\forall x, y \in X, d_{Y}(f(x), f(y)) \leq d_{X}(x, y)
	\end{equation*}

	Si $k \in [0, 1[$, on dit que $f$  est \textbf{strictement contractante}.
\end{definition}

\begin{theorem} [du point fixe de Banach ou méthode des approximations
	successives]
	\label{theorem:point_fixe_banach}
	Soit $(X, d)$ un espace métrique complet.

	Soit $\GSfunction{f}{(X, d)}{(X, d)}$ une fonction strictement contractante.

	Alors $f$ admet un unique point fixe $x^{*}$.
	De plus, si $a \in X$, alors la suite de terme initiale $x_{0} = a$ et $x_{n
	+ 1} = f(x_{n})$ tend vers $x^{*}$.
\end{theorem}

\ifdefined\outputproof
\begin{proof}

\end{proof}
\fi

% Page 31 Sam.
%\begin{theorem}
	%Soit $(X, d)$ un espace métrique et $Y$ un espace topologique.
%\end{theorem}

\begin{theorem}
	\label{theorem:inversion_globale}
	Soient $E, F$ deux espaces de Banach.

	Soit $\mathcal{U}$ un ouvert de $E$.

	Soit $\GSfunction{f}{\mathcal{U}}{F}$ une application de classe
	$\mathcal{C}^{1}$ (ie continûment différentiable).

	Alors, les assertions suivantes sont équivalentes.

	\begin{itemize}
		\item $f$ est un $\mathcal{C}^{1}$-difféomorphisme de $\mathcal{U}$ sur
			un ouvert de $F$.
		\item $f$ est injective et $\forall x \in \mathcal{U}$, $df(x) \in
			\GSisomorphismeHomo{E}{F}$
	\end{itemize}
\end{theorem}

\ifdefined\outputproof
\begin{proof}

\end{proof}
\fi

% Remarques page 32 et 33 Sam

Donnons une application de ces deux théorèmes.

\begin{exemple}
	Il n'existe pas de sous-groupe de $\inversibleMatrixSpace{n}{\real}$
	arbitrairement petit. C'est-à-dire qu'il existe un voisinage $\mathcal{V}$ ouvert de
	$1_{n}$ dans $\inversibleMatrixSpace{n}{\real}$ tel que si $G$ est un
	sous-groupe de $\inversibleMatrixSpace{n}{\real}$ avec $G \subseteq
	\mathcal{V}$, alors $G = \GSset{1_{n}}$.
\end{exemple}
