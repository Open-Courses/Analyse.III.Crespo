\chapter{Applications différentiables}

\section{Motivation}
	Étant donné une application entre deux espaces vectoriels, on veut pouvoir
	approcher l'image de la fonction sur un interval grace à une fonction
	affine, facile à calculer.

\section{Définition et propriétés}

\section{Règles de calcul}

addition

soustraction

multiplication

composée

\section{Différentielle de l'application inverse}

\section{Différentielle composante}

\section{Différentielle des applications multilinéaires}

\section{Formule de Leibniz}

\section{Dérivée directionnelle et dérivée partielle}
