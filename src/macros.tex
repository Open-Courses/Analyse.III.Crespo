\usepackage{amsmath}


%GSsequence :
%		#1 : represention of elements of the sequences
%		#2 : indices
%		#3 : set definition
\newcommand{\GSsequence}[3]{$(#1_{#2})_{#2 \in #3}$}

%GSset :
%		#1 : global set
%		#2 : condition
\newcommand{\GSset}[2]{$\left\{#1 \, | \, #2 \right\}$}

%GSprodSet :
%		#1 : indice
%		#2 : begin indice
%		#3 : end indice
%		#4 : set
\newcommand{\GSprodSet}[4]{$\displaystyle \prod_{#1 = #2}^{#3} #4_{#1}$}

%GSsum :
%		#1 : indice
%		#2 : begin indice
%		#3 : end indice
%		#4 : element
\newcommand{\GSsum}[4]{$\displaystyle \sum_{#1 = #2}^{#3}$ #4}

\newcommand{\GSintervalCC}[2]{$\left[#1, #2\right]$}

%Analysis :

%GSApplication :
%       #1 : name funtion
%       #2 : begin set
%       #3 : end set
\newcommand{\GSfunction}[3]{#1 : #2 $\rightarrow$ #3}
%GSnorme :
%		#1 : elements which norme is applied on
\newcommand{\GSnorme}[1]{$||#1||$}

%GSnormeDef :
%		#1 : elements which norme is applied on
%		#2 : norme indice
\newcommand{\GSnormeDef}[2]{$||#1||_{#2}$}

%GSnormedSpace :
%		#1 : vectorial space
%		#2 : \GSnorme[Def] with dot as element.
\newcommand{\GSnormedSpace}[2]{$($#1, #2$)$}

%GSdual
%		#1 : vectorial space
\newcommand{\GSdual}[1]{#1^{*}}

%GSbidual
%		#1 : vectorial space
\newcommand{\GSbidual}[1]{#1^{**}}

%GSendomorphism
\newcommand{\GSendomorphism}[1]{End(#1)}

%GShomomorphisme
\newcommand{\GShomomorphisme}[2]{Hom(#1, #2)}

%GScontinueEndo
\newcommand{\GScontinueEndo}[1]{$\mathcal{L}(#1)$}

%\GScontinueHomo
\newcommand{\GScontinueHomo}[2]{$\mathcal{L}(#1, #2)$}
